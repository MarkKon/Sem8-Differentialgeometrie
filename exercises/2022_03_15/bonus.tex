\documentclass[11pt]{article}
\usepackage{ {util/personalmacros} }
\usetikzlibrary{arrows,calc, angles ,patterns}

\title{Exercise 2 Bonus} 

\author{Konstantin Mark\\
Differential Geometry\\ 
\textsc{TU Wien}
}
\date{17.03.2022}


\begin{document}
\maketitle


\section*{5**}
{\itshape
The pair of curves $\gamma,\gamma^\epsilon$ from Exercise 4 can be viewed as the tracks of the rear and front wheel of a bicycle (the bycicle frame is always tangent to the trajectory of the rear wheel, while the front wheel can steer).\\
Given the tracks of the rear and front wheel, can you tell which way the bicycle has travelled?
}\\~\\

\subsection{Direction of travel}

\begin{remark}
First of all some discussion on the notion of \textit{direction of travel} is necessary. Clearly there is no definition of \textit{forward travel}, however we can compare the direction of travel of two curves. This leads to the two following two definitions.
\end{remark}
\begin{definition}
Let $\gamma:I\to \mathbb R^n, \delta: J\to \mathbb R^n$ be two different parametrizations of the same curve. Then we define the pair $(\gamma, \delta)$ to be \textbf{unidirectional} if there is a reparametrization $\varphi: I\to J$ that is monotone increasing.
\end{definition}
\begin{definition}
Let $\gamma:I\to \mathbb R^n, \delta: J\to \mathbb R^n$ be two different parametrizations of the same curve. Then we define the pair $(\gamma, \delta)$ to be \textbf{contrary} if there is a reparametrization $\varphi: I\to J$ that is monotone decreasing.
\end{definition}
\begin{lemma}
All pairs $(\gamma,\delta)$ of unit-speed parametrizations of a curve are either unidirectional or contrary.
\end{lemma}
\begin{proof}
Let  $\gamma:I\to \mathbb R^n, \delta: J\to \mathbb R^n$ be unit speed and $\varphi: I\to J$ a reparametrization. Then by the chain rule $$1 =\norm{\frac{\mathrm d}{\mathrm ds}\delta\circ\varphi(s)} = \norm{\dot\delta\circ \varphi(s)}\cdot|\dot\varphi(s)| = |\dot\varphi(s)|.$$
Thus $\dot\varphi(s)$ is either $-1$ or $1$. Smoothness guarantees that it has to take the same value over all values of $s$. Thus $\varphi$ has either the form $a+s$ or the form $a-s$ for some $a\in \mathbb R$.
\end{proof}
\begin{lemma}
\label{lem:bic_dir}Let $\gamma$ be a regular curve in $\mathbb R^n$, $\epsilon\in \mathbb R$ and define the function $\sigma: x\mapsto -x$. Then the pair $(\gamma^\epsilon, (\gamma\circ\sigma)^{-\epsilon})$ parametrizes the same curve and forms a contrary pair.
\end{lemma}
\begin{proof}
\begin{equation*}
    (\gamma\circ\sigma)^{-\epsilon}(-t) = \gamma\circ\sigma(-t) -\epsilon\frac{\mathrm d}{\mathrm ds}(\gamma\circ\sigma(-t)) = \gamma(t) + \epsilon\dot\gamma(t)
\end{equation*}and thus $(\gamma\circ\sigma)^{-\epsilon}) = \gamma^\epsilon\circ\sigma$. Clearly $\sigma$ is monotone decreasing and the pair is contrary.
\end{proof}

\begin{remark}
Lemma \ref{lem:bic_dir} shows that to have any proper notion of direction of travel for the bicycle curve we must fix the values for $\epsilon$ to be nonnegative. As $\epsilon = 0$ clearly is not interesting when it comes to direction of travel (ie. any bicycle curve with $\epsilon = 0$ is just the curve itself) from now on, we shall restrict the values of $\epsilon$ to positive real numbers.
\end{remark}

\subsection{Reformulating the problem}

\begin{remark}
Let us reformulate the goal: Given two curves $\gamma, \delta$, where we know that there is either $\epsilon>0$ with $\delta = \gamma+\epsilon \dot\gamma$ or $\epsilon'>0$ with $\gamma = \delta+\epsilon'\dot\delta$ (with the possibility that there are both, or also multiple such formulations), can we conclude that there is no contrary reparametrization of the same curve?
\end{remark}
\begin{definition}\label{def:bicycle}
Let $\gamma:I\to \mathbb R^n, \delta: J\to \mathbb R^n$ be two curves. A tuple $(\gamma, \delta, \epsilon, \pm1)$ is said to be a \textbf{bicycle} of the curves $(\gamma, \delta)$ if $\epsilon > 0$ and there are $\varphi, \psi$ monotone and with the same monotonicity with $\delta\circ \psi = (\gamma\circ \varphi)^\epsilon$. $\gamma$ is said to be the \textbf{rear wheel}, $\delta$ is said to be the \textbf{front wheel}. If $\varphi, \psi$ are both monotone increasing, we say the bicycle is \textbf{canonical} and set the last tuple value to $1$, if they are both monotone decreasing, we say the bicycle is \textbf{contrarian} and set the last tuple value to $-1$.
\end{definition}
\begin{definition}
Define $B(\gamma, \delta)$
the\textbf{ set of bicycles} with rear wheel $\gamma$ and front wheel $\delta$.
\end{definition}
\begin{remark}
Clearly our goal is to find out if $\#|B(\gamma,\delta)| = 1$ when given curves $\gamma,\delta$. Clearly there are curves with $\#|B(\gamma,\delta)| = 0$ (this might even happen if the front wheel curve is parametrized "contrary" to the rear wheel curve). The following section will show that there are curves with $\#|B(\gamma,\delta)| = 2$ and the statement $\#|B(\gamma,\delta)|\leq 1$ posed in the exercise cannot be made in general.
\end{remark}
\begin{remark}\label{rem:alg}
From Definition \ref{def:bicycle} it is evident how we may find out the direction of travel: Look at the tangents of the curves. A necessary requirement for $\gamma$ to be the rear wheel is that its tangent intercepts $\delta$ at a constant distance. 
\end{remark}



\subsection{Examples of ambiguity in direction}

\begin{example}[Cycling in a circle]\label{exa:circle}
Consider the two circles given by the curves \begin{align*}
\gamma:&
    \left\{
    \begin{array}{ccl}
        \mathbb R& \to & \mathbb R^2 \\
        t & \mapsto & (\cos(t),\sin(t))^T
     \end{array}
     \right.
     \\
\delta:&
    \left\{
    \begin{array}{ccl}
        \mathbb R & \to & \mathbb R^2 \\
        t & \mapsto & 2(\cos(t),\sin(t))^T
     \end{array}
     \right.
\end{align*}
 shown in Figure \ref{fig:circles}. The tangent to $\gamma(t)$ intersects $\delta(\mathbb R)$ at points $d:= \pm \sqrt{2^2-1^2} = \pm \sqrt3$ away, but with an angle shift of $\varphi := \pm \arccos\frac{\sqrt3}2 = \frac\pi6$. That is, the curves $\gamma, \delta$ satisfy $$\delta(t+\varphi) = \gamma(t) + d\dot\gamma(t).$$ However, again with the function $\sigma: x\mapsto -x$ it also holds that $$(\delta\circ \sigma)(t+\varphi) = (\gamma\circ \sigma)(t) + d (\gamma\circ\sigma)^\cdot(t).$$ Therefore we have $B(\gamma, \delta)= \{(\gamma, \delta, d,1), (\gamma, \delta, d,-1)\}$ and there is both a canonical and a contrarian bicycle.
\end{example}

\begin{figure}[!h]
    \centering

    
\begin{tikzpicture}[scale = 2]
\def\y{3};
\coordinate (center) at (2,2);
\coordinate (lowercenter) at (2,3);
\coordinate (left) at ($(2,3) - sqrt(\y)*(1,0)$);
\coordinate (right) at ($(2,3) + sqrt(\y)*(1,0)$);



\draw[thick, gray] (center) circle (2);
\draw[thick, gray] (center) circle (1);
\draw (center) -- (lowercenter); 
\draw (center) -- (right) ;
\draw[red] (left) -- (right);

\fill (center) circle (1pt);
\fill (lowercenter) circle (1pt);
\fill (left) circle (1pt);
\fill (right) circle (1pt);

\draw (1,3.2) node {$d = \sqrt{3}$};
\draw (3,3.2) node {$d = \sqrt3$};
\draw (1.85,2.5) node {$1$};
\draw (3.3, 2.6) node {$2$};
\draw (3,1.5) node {$\gamma$};
\draw (3,.5) node {$\delta$};

\pic [draw, -, "$\varphi$", angle eccentricity=1.5] {angle = right--center--lowercenter};

\end{tikzpicture}

    \caption{Curves where determination of direction of travel is impossible. The tangent to $\gamma$ has constant-length intersects with $\delta$ at two different points.}
    \label{fig:circles}
\end{figure}

\begin{remark}
Due to the periodicity of the curves, there are many different ways to choose the functions $\varphi, \psi$ in Definition \ref{def:bicycle}, but the definition guarantees that, bar the direction of travel, there is no difference in bicycle by how fast or at what time a certain point is traversed.
\end{remark}
\begin{remark}
If, in Example \ref{exa:circle} we instead choose the curves
\begin{align*}
\gamma:&
    \left\{
    \begin{array}{ccl}
        [0,2\pi)& \to & \mathbb R^2 \\
        t & \mapsto & (\cos(t),\sin(t))^T
     \end{array}
     \right.
     \\
\delta:&
    \left\{
    \begin{array}{ccl}
        [\frac\pi6,2\pi+\frac\pi6) & \to & \mathbb R^2 \\
        t & \mapsto & 2(\cos(t),\sin(t))^T
     \end{array}
     \right.
\end{align*}
then we could only find a canonical bicycle, since starting and endpoints must align.
\end{remark}
\begin{example}[Cycling in a straight line]\label{exa:straight}
Consider $\gamma= \delta$ a straight line. Then $$B(\gamma,\delta) = \{(\gamma, \delta,\epsilon, 1): \epsilon>0\}\cup\{(\gamma,\delta,\epsilon,-1), \epsilon >0\}$$ and there is an infinite number of bicycles.
\end{example}
\begin{remark}
To summarise: The "algorithm" described in \ref{rem:alg} can fail (if the curves are not generated by a bicycle) or produce multiple bicycles in different directions, as seen in Examples \ref{exa:circle} and \ref{exa:straight}. Clearly, these examples work for any dimensions $\geq 2$ as the circle in Example \ref{exa:circle} can be embedded into any $\mathbb R^n$. In dimension $1$, regularity of curves guarantees that any regular curve does not change direction, but there are, as seen in Example \ref{exa:straight} infinitely many bicycles if the curve is infinitely long.
\end{remark}
\end{document} 